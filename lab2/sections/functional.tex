% !TeX root = ../note.tex
\section{Функциональные требования}

\subsection{Описание данных}

\subsubsection{Тарифы}
Тарифы, предоставляемые оператором. О каждом тарифе хранятся следующие данные:
\begin{itemize}
    \item название
    \item краткое и полное описания
    \item предоплата
    \item абонентская плата
    \item ежемесячный трафик
    \item ежемесячные минуты внутри сети
    \item ежемесячные минуты в другие сети
    \item цена интернет-трафика
    \item цена видеосвязи
    \item цена SMS
    \item цена MMS
    \item цена международной связи
    \item архивность (то есть активен ли тариф для заказа)
    \item акционная абонентская плата
    \item акционная цена интернет-трафика
    \item время действии акции
    \item дополнительная информация - для хранения специфических данных
\end{itemize}

\subsubsection{Дополнительные услуги}
\begin{itemize}
    \item название услуги
    \item краткое и полное описания
    \item цена подключения
    \item периодичность (одноразовая или ежемесячная услуга)
    \item архивность (то есть активна ли услуга для заказа)
    \item ограничения - только для хранения, обрабатываются на back-end
    \item дополнительная информация - для хранения специфических данных
\end{itemize}

\subsubsection{Клиенты}
Общая информация о клиентах оператора
\begin{itemize}
    \item телефонный номер
    \item имя
    \item фамилия
    \item отчество/второе имя/т.п., в зависимости от паспортных данных клиента
    \item номер паспорта
    \item прописка
    \item подключенный тариф и время подключения тарифа
    \item подключенные услуги и время подключения каждой услуги
    \item текущий баланс
    \item остаток интернет-трафика
    \item остаток минут внутри сети
    \item остаток минут в другие сети
\end{itemize}

\subsubsection{Партнёры}
Все операторы сотовой связи сотрудничают с компаниями-производителями телефонов и аксессуаров.
\begin{itemize}
    \item название компании-партнёра
    \item веб-сайт
    \item описание компании
\end{itemize}

\subsubsection{Товары партнёров}
Так как полная информация о каждом товаре хранится у поставщика, то в данном проекте будет хранится лишь краткая информация о товаре.
\begin{itemize}
    \item название
    \item категория
    \item партнёр-поставщик
    \item цена товара
    \item доступность на складе
    \item акционная цена
    \item время действии акции
\end{itemize}

\subsubsection{Заказы клиентов}
Данные по каждому товару, купленному каждым клиентом
\begin{itemize}
    \item клиент
    \item товары и их цены
    \item время заказа
    \item общая цена заказа
\end{itemize}

\newpage

\subsection{Описание API}

\subsubsection{Тарифы}
\begin{itemize}
    \item создание/обновление информации
    \item получение информации о тарифах с помощью фильтров
    \item добавление/получение/обновление/удаление акции
    \item перевод в архивные
\end{itemize}

При переводе тарифа в архивные, обновление информации о нём должно быть запрещено.

\subsubsection{Дополнительные услуги}
\begin{itemize}
    \item создание/получение/обновление информации
    \item перевод в архивные
\end{itemize}

При переводе услуги в архивные, обновление информации о ней должно быть запрещено.

\subsubsection{Клиенты и заказы}
\begin{itemize}
    \item добавление/получение/обновление/удаление информации
    \item создание заказа
    \item получение информации о заказах клиента
    \item получение информации об услугах клиента
\end{itemize}

\subsubsection{Партнёры и товары}
\begin{itemize}
    \item добавление/получение/обновление/удаление информации о партнёре
    \item добавление/получение/обновление/удаление информации о товаре (возможность одновременного создания нескольких товаров)
\end{itemize}

\newpage
